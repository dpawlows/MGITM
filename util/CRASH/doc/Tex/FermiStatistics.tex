%  Copyright (C) 2002 Regents of the University of Michigan, portions used with permission 
%  For more information, see http://csem.engin.umich.edu/tools/swmf

\section{Effect of the Fermi statistics on Thermal Ionization}

While simulating the ionization equilibrium in partially ionized electron-ion plasmas using Saha 
equations, the electrons are usually assumed to be an ideal Boltzmann gas.  However, the electron is a Fermion with spin of 1/2. In practical calculations the conditions for
applicability of the Boltzmann gas model for electrons are often not satisfied, resulting in
very low accuracy. Therefore, it is worth while checking whether or not one needs to assume Boltzmann statistics for the electrons when solving the ionization equlibrium.

It appears that for realistic quantitative simulations this assumption is completly 
unnecessary, as solving the ionization equilibrium under the incorrect assumption of 
Boltzmann statistics is not any easier than that with the correct Fermi statistics for 
electrons.

In a partially ionized plasma the free electron density, corrected by the effects of the Fermi statistics 
('the exchange interaction'), should be used in solving the Equation-Of-State (EOS), which is also 
directly affected by the exchange interactions. There is no need to remind that 
{\it at the given electron density} the exchange interaction increases the electron pressure and the 
internal energy density (see \cite{ll}). On the other hand, in partially ionized plasmas this effect may 
be partially or even fully balanced by the electron density decrease due to the exchange interaction 
effect on the ionization equilibrium.  

% Helmholtz free energy via N_i
{\bf Helmholtz free energy.} Consider an ionized monatomic gas with positive non-complex ions. The Helmholtz free energy,
$F=F_{ion}+F_e$, is assumed to be the total of contributions from each of the ion charge states,
$i=0$ to $i_{\max}$ (we apply Eq.(42.3) from \cite{ll} to account for these contributions), as well as the contribution from electrons:
\begin{equation}\label{freeenergy}
F=-T
\sum_{i=0}^{i_{max}}{
N_i\log\left[g_i
             \frac{eV}{N_i}\left(\frac{MT}{2\pi \hbar^2}\right)^{3/2}\exp \left(-\sum_{j=0}^{i-1}\frac{I_j}T \right)\right]}+F_e,
\end{equation}  
where $N_i=n_iV$ is the total number of ions in the charge state $i$ in the volume V and
$I_j$ ($j = 0, 1, 2, \dots$) is the energy needed to ionize an atom or ion from the charge state $j$ to the charge state $j+1$ (the ionization potential), and $g_i$ is a 
statistical weight of an ion (atom) in a given charge state 
(see Section IV for more detail).

{\bf Ionization equilibrium: formulation of the problem.} Now we formulate the requirement for the
ionization equilibrium with respect to the reaction $(i)\leftrightarrow(i+1)+e$ for each ion charge state, $i$.
The Helmholtz free energy is a minimum at the equilibrium set of $N_i$ and $N_e$. Therefore, the total
derivative of F with respect to $N_e$ should be zero:
\begin{equation}
\frac{\partial F}{\partial N_e} + \frac{d N_i}{d N_e} \frac{\partial F}{\partial N_i} +
\frac{d N_{i+1}}{d N_e} \frac{\partial F}{\partial N_{i+1}} = 0.
\end{equation}
For the reaction under
consideration, the increments in the particle numbers should be 
related as follows: $dN_{i}=-dN_{i+1}=-dN_e$.
Therefore the requirement, $dF/dN_{i}-dF/dN_{i+1}-dF/dN_e=0$, gives:
\begin{equation}\label{equili}
-T\log\left[\frac{g_{i}}{N_{i}} e^{-\sum_{j=0}^{i-1}I_j/T}\right] + T\log\left[\frac{g_{i+1}}{N_{i+1}} e^{-\sum_{j=0}^{i}I_j/T}\right]-\mu_e=0,
\end{equation}
where we applied the definition of the chemical potential, $\mu=(\partial F/\partial N)_{T,V}$, to the electron gas.
The solution of the ionization equilibrium, therefore, reads:
\begin{equation}
N_{i+1}/g_{i+1}=(N_i/g_i)e^{(-\mu_e-I_i)/T},
\end{equation}
or, applying this recursively:
\begin{equation}\label{saharecursive}
N_{i}/g_{i}=(N_0/g_0)e^{(-i\mu_e-\sum_{j=0}^{i-1}I_j)/T}=(N_0/g_0)(g_e)^ie^{-\sum_{j=0}^{i-1}I_j/T},\,\,\,g_e=e^{-\mu_e/T},
\end{equation}
where $g_e$ is the effective statistical weight of a free electron.
Indeed, the effective statistical weight of i electrons combined with an ion in the charge state, i, is the
product of statistical weights for each of the particles under consideration, $(g_e)^i g_i$, in accordance with Eq.(\ref{saharecursive}).

In the limiting case of $g_e\gg1$ 
(a Boltzmann gas of electrons with large negative value of $\mu_e$), $g_e$ might be interpreted as the
large number of elementary quantum states the detached electron can occupy, which facilitates the ionization by resulting in a higher total probability for 
the ionized state. In the opposite limiting case of a degenerate Fermi gas of electrons, the positive chemical potential, $\mu_e>0$, 
tends to the Fermi energy $E_F$, which in
this limiting case is much greater than the temperature. Accordingly, the exponentially low value of $g_e=e^{-E_F/T}$ in this case means the low 
probability for an electron to jump from a bound state with negative energy to a free state above the threshold of the positive Fermi energy.  

{\bf Partition function and electron density}. We now introduce the ion partition function, $p_i=N_i/N_a$, $N_a$, being 
the total number of atoms. Since the partition function is normalized by unity,
we have:
\begin{equation}
p_i=\frac{g_i(g_e)^ie^{-E_i/T}}S,
\end{equation}
where we introduced the statistical sum:
\begin{equation}
S=\sum_{i=0}^{i_{max}}\left[g_i(g_e)^ie^{-E_i/T}\right],
\end{equation}
as well as the total ionization energy spent to ionize the atom to the state $i$:
\begin{equation}
E_i=\sum_{j=0}^{i-1}I_j.
\end{equation}
Introducing the averaging operator acting on an arbitrary function of the ion charge number, $\langle f_i\rangle=\sum p_if_i$, and assuming quasi-neutrality, $N_e = \sum{i N_i}$, we
obtain the expression for the electron density:
\begin{equation}\label{zavr}
Z=N_e/N_a=\langle i\rangle.
\end{equation}
On the other hand, for given $T$ and $n_a=N_a/V$ the electron concentration may be found as a function of $g_e(=e^{-\mu_e/T})$.
Now we assume electrons form an ideal Fermi gas.
This assumption immediately gives us another relationship between the electron density and $g_e$ (see Eq.(56.5) from \cite{ll}):
\begin{equation}\label{zfe}
Z=g_{e1}{\rm Fe}_{1/2}(g_e).
\end{equation}
The coupled equations (\ref{zavr}) and (\ref{zfe}) are used below to solve $Z$ and $g_e$. Here
\begin{equation}
g_{e1}(T,N_a/V)=\frac{2V}{N_a}\left(\frac{m_eT}{2\pi \hbar^2}\right)^{3/2}
\end{equation}
is a value such that in the {\it Boltzmann} electron gas $g_e = g_{e1}/Z$ would hold. ${\rm Fe}_\nu(g_e)$ is the Fermi function:
\begin{equation}
{\rm Fe}_\nu(g_e)=\frac1{\Gamma(\nu+1)}\int{\frac{x^\nu dx}{g_ee^x+1}},
\end{equation}
where $\Gamma$-function is introduced as usually: $\Gamma(\nu+1)=\nu \Gamma(\nu),\,\Gamma(1/2)=\pi^{1/2}$.
Below we use the following auxiliary functions:
\begin{equation}
R^-(g_e)=\frac{{\rm Fe}_{-1/2}(g_e)}{{\rm Fe}_{1/2}(g_e)}, \qquad
R^+(g_e)=\frac{{\rm Fe}_{3/2}(g_e)}{{\rm Fe}_{1/2}(g_e)}.
\end{equation}
Detailed discussion on the Fermi functions is delegated to Appendix A.

{\bf Derivatives along the ionization curve.}
Taking differential of the equation of ionization equilibrium, $G=0$, one gets an equation relating the differentials
of different variables along the curve of ionization equilibrium:
\begin{equation}\label{diffge}
\frac{dg_e}
{g_e}
\left(
\langle i^2\rangle-Z^2+ZR^-(g_e)
\right)+\frac{dT}{T^2}
\left(\langle iE_i\rangle-\langle E_i\rangle Z)\right)=\left(\frac32\frac{dT}{T}+\frac{dV}V\right)Z,
\end{equation}
here we substitute wherever possible $Z$ for $\langle i\rangle$ and $g_{e1}{\rm Fe}_{1/2}(g_e)$.
Accordingly, the differential of $Z$ is:
\begin{equation}\label{diffz}
dZ=\left(\frac32\frac{dT}{T}+\frac{dV}V-\frac{R^-(g_e)dg_e}{g_e}\right)Z.
\end{equation}

{\bf To solve the ionization equilibrium} for the given $T$ and $n_a=N_a/V$, one needs to solve $g_e$ from equation $G(g_e)=0$, 
$G(g_e)=\langle i\rangle-g_{e1}{\rm Fe}_{1/2}(g_e)$, which is obtained by means of excluding $Z$ from Eqs.(\ref{zavr},\ref{zfe}). 
It may be solved using the Newton-Rapson iterations with any trial value, $\log(g_e)_{old}$, the improved value, $\log(g_e)_{new}$, 
is obtained from the equation as follows:
\begin{equation}\label{iter}
\log(g_e)_{new}=%\log(g_e)_{old}-\frac{G((g_e)_{old})}{G^\prime((g_e)_{old})}=
\log(g_e)_{old}-\frac{\langle i\rangle-g_{e1}{\rm Fe}_{1/2}((g_e)_{old})}{
\langle i^2\rangle-\langle i\rangle^2+ Z R^-((g_e)_{old}) },
\end{equation}
where the derivative $g_eG^\prime$, which should stand in the denominator of 
Eq.(\ref{iter}), is derived using the easy-to-check equations as follows:
\begin{equation}
g_e\frac{d{\rm Fe}_\nu(g_e)}{dg_e}=-{\rm Fe}_{\nu-1}(g_e),
\end{equation}
and for any set of values in the charge states, $f_i$:
\begin{equation}
g_e\frac{\partial\langle f_i\rangle}{\partial g_e}=\langle if_i\rangle-\langle f_i\rangle\langle i\rangle,
\end{equation}
which is a particular case of Eq.(\ref{differmean}).

We see that to solve iterations in Eq.(\ref{iter}) for Fermi gas of electrons is not a
more computationally intense problem compared to the same problem assuming electrons 
to be a Boltzmann gas. In the latter case, $g_e \to +\infty$, we have ${\rm Fe}_\nu(g_e) \approx 1/g_e$
(see Eq.(\ref{feboltzmann}) below) and $g_e \approx g_{e1}/Z$, which allows us to iterate Eq.(\ref{iter})
as a somewhat simpler equation for $Z$:
$\log (Z_{new})=\log(Z_{old})+(\langle i\rangle-Z_{old})/(Z_{old}+\langle i^2\rangle-\langle i\rangle^2)$. However in any case, the most cumbersome computations 
while solving Eq.(\ref{iter}) for a Fermi gas, or the equation for $Z$ for a Boltzmann gas, is in explicitly calculating the 
numerous partition functions for many charge
states and excitation levels. Compared with these bulk computations, the presence of the Fermi functions in Eq.(\ref{iter}), which may be tabulated 
for all interesting cases of $\nu=-1/2,1/2,3/2$, % as well as a single inverse function, ${\rm Fe}^{-1}_{1/2}$ 
does not matter at all.

Therefore, we 
do not see any reason for applying the assumption of a Boltzmann electron gas in modelling the ionization equilibrium
in real dense plasmas.

{\bf Plasma thermodynamics and Equation-Of-State.} Now we substitute the ion partition function into Eq.(\ref{freeenergy}). After some algebra we obtain:
\begin{equation}\label{equile}
F = -TN_a\log\left[\frac{eV}{N_a}\left(\frac{MT}{2\pi \hbar^2}\right)^{3/2}\right]-TN_a\log S +\Omega_e,
\end{equation}
the thermodynamic potential $\Omega_e=F_e -\mu_eZN_a$ for Fermi gas of electrons is given by Eq.(56.6) from \cite{ll}:
\begin{equation}
\Omega_e = -[g_{e1}N_a]T{\rm Fe}_{3/2}(g_e),
\end{equation}
the product in the square brackets, being independent of $N_a$ because $g_{e1}\sim N_a^{-1}$.  Eq.(\ref{equile}) provides the free energy in the case of
local thermodynamic equilibrium with the first term being the contribution from the ion translational energy. This term may be written as the function of the ion 
temperature, in the case the latter differs from the electron temperature. Unless the ion-ion interaction is taken into account, this first term gives the contributions 
of $n_aT_i$ and $3n_aT_i/2$ to the total plasma pressure and total energy density correspondingly. The second term is
the Boltzmann distribution of ions over the ionization and excitation states, expressed in terms of the statistical sum. Finally, the electron gas with the
variable particle number gives the contribution of $\Omega_e$ instead of $F_e$. 

While differentiating Eq.(\ref{equile}) with respect to $T$ and $V$, it is important that the derivatives
by $g_e$ from the second and third terms cancel 
each other: $g_e(\partial \log S/\partial g_e)=\langle i\rangle=Z$ and $-g_eg_{e1}{\rm Fe}^\prime_{3/2}(g_e)=g_{e1}{\rm Fe}_{1/2}=Z$. That is why for the internal energy density, 
${\cal E}$, and for the pressure we find:
\begin{equation}
{\cal E} = -\frac{T^2}V\left(\frac{\partial}{\partial T}\left(\frac F T\right)\right)=
{\cal E}_i+{\cal E}_e,\qquad{\cal E}_i=\frac32Tn_a,
\qquad{\cal E}_e=n_a\left[\frac32TZR^+(g_e)+\langle E_i\rangle\right],
\end{equation}
\begin{equation}
P = -\frac{\partial F}{\partial V}=P_i+P_e,\qquad
P_i = n_aT,\qquad
P_e = n_aTZR^+(g_e).
\end{equation}

However, while calculating the second order thermodynamic derivatives, like the specific heat, the derivatives of $g_e$ essentially sophisticate the 
calculations. The result may be expressed in terms of covariances:
$\langle\delta^2 i   \rangle=\langle(i-Z)^2\rangle$,
$\langle\delta^2 E_i \rangle=\langle(E_i-\langle E_i\rangle)^2 \rangle$ and 
$\langle\delta i\delta E_i \rangle=\langle(E_i-\langle E_i\rangle)(i-Z)\rangle$.
In a similar way one can find the specific heat in an isochoric process, per the unit of volume:
\begin{equation}
C_{Ve}=\frac{\partial {\cal E}_e}{\partial T}=n_a\left[\frac{\langle\delta^2 E_i \rangle}{T^2}+\frac{15}4ZR^+
-\frac{\left(\frac32Z-\frac{\langle\delta E_i \delta i\rangle}T\right)^2}{\langle\delta^2i\rangle+ZR^-}\right],
\end{equation}
the temperature derivative of pressure:
\begin{equation}
\frac {\partial P_e}{\partial T} = n_aZ\left[\frac52 R^+ -
\frac{\frac32Z-\frac{\langle\delta E_i \delta i\rangle}T}{\langle\delta^2i\rangle+ZR^-}\right],
\end{equation}
as well as the isothermal compressibility:
\begin{equation}
V\frac{\partial P_e}{\partial V}=-\frac{Z^2n_aT_e}{\langle\delta^2i\rangle+ZR^-}.
\end{equation}
For simplicity in the above equations, the contributions due to ion translational motions,
\begin{equation}
C_{Vi}=\frac32n_a, \qquad
\frac{\partial P_i}{\partial T}=n_a, \qquad
V\frac{\partial P_i}{\partial V}=-n_aT,
\end{equation}
are omitted.

The speed of sound, $C_s$, is defined in terms of the adiabatic comressibility (at constant entropy), $C_s^2=\left(\frac{\partial P}{\partial \rho}\right)_{\rm ad}$, which
may be parametrized in terms of effective adiabatic index, $\gamma$, such that 
$\gamma\frac{P}\rho=\left(\frac{\partial P}{\partial \rho}\right)_{\rm ad}$, herewith $\rho$ is the mass density. To calculate this, one can take Eq.(3.72) from 
\cite{drake}, 
$$
\left(\frac{\partial P}{\partial \rho}\right)_{\rm ad}=\left(\frac{\partial P}{\partial \rho}\right)_T-
\frac{\rho}{C_V}\left[
\left( \frac{\partial({\cal E}/\rho)}{\partial \rho}\right)_T-\frac{P}{\rho^2}\right]\left(\frac{\partial P}{\partial \rho}\right)_T.
$$
Note that in \cite{drake} both the internal energy and the specific heat are related per a unit of mass, while we relate them to the unit of volume.  
Now we apply the thermodynamic identity as follows:
\begin{equation}
\left( \frac{\partial ({\cal E}/\rho)}{\partial \rho} \right)_{T} - \frac{P}{\rho^2} =
-\frac{T}{\rho^2} \left( \frac{\partial P}{\partial T} \right)_{\rho},
\end{equation}
which gives:
\begin{equation}
\gamma = \frac{\rho}{P} \left( \frac{\partial P}{\partial \rho} \right)_T +
\left( \frac{\partial P}{\partial T} \right)^2_\rho \frac{T}{C_V P}.
\end{equation}
Note that in the last three equations we denote the derivative at constant $V$ as that at constant $\rho$ and used the derivatives over $\rho$ instead of those over $V$:
 $V\frac\partial{\partial V}=-\rho\frac\partial{\partial \rho}$.

{\bf Discussion:} Estimating the effect of the Fermi statistics on the ionization degree.
Eqs.(\ref{diffge},\ref{diffz}) allow us to evaluate the effect of electron Fermi statistics on ionization.
From Eq.(\ref{feboltzmann}) one can
see that for large $g_e$ (Boltzmann gas) the equation, $G(g_e)=0$, reduces to $\langle i\rangle-g_{e1}/g_e=\delta_{\rm Fe}$,
where $\delta_{\rm Fe}$ is a small negative correction to the Fermi function at $g_e \to \infty$:
\begin{equation}\label{feboltzmann}
{\rm Fe}_\nu(g_e)=\frac{1}{\Gamma(\nu+1)} \int \frac{x^\nu dx}{g_e e^x + 1} =
\frac{1}{\Gamma(\nu+1)} \int \frac{x^\nu dx}{g_e e^x} + \delta_{\rm Fe} = \frac{1}{g_e} + \delta_{\rm Fe}.
\end{equation}
%Considering $\delta_{\rm Fe}$ negligible, one can find $g_e=g_{e1}/Z$ from Eq.(\ref{zfe})

Assuming $\delta_{\rm Fe}$ to be a small increment in the right hand side of Eqs.(\ref{diffge},\ref{diffz}), and by
finding $dg_e/\delta_{\rm Fe}$ from Eq.(\ref{diffge}) at $dV=dT=0$ and then finding $dZ/\delta_{\rm Fe}$ from Eq.(\ref{diffz}) gives:
\begin{equation}
\delta Z=Z\delta_{\rm Fe}\frac{\langle i^2\rangle-Z^2}{\langle i^2\rangle-Z^2+Z}. 
\end{equation}
The correction is negative as long as $\delta_{\rm Fe}$ is negative. Thus, due to the Fermi gas effects for detached electrons, the ionization degree is
always lower than that predicted by the Saha equilibrium equations under the assumption of a Boltzmann electron gas.

This effect should be accounted for while
treating the effect of the Fermi statistics for electrons in the equation of state. Specifically, {\it at constant electron density} the exchange interactions 
between the electrons increases the electron pressure, but in the partially ionized plasma the magnitude (if not the sign!) of this effect can be  
compromised by the pressure reduction due to the decrease in electron density. 


